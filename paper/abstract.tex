\begin{abstract}
This paper is chiefly concerned with the analysis of tumblr, an online 
social network that appears to exhibit several properties that make it 
distinct from other popular online social networks (hereafter referred 
to as OSNs), such as Twitter, Facebook, Pinterest, and Instagram.  These 
differences range from changes in user/content creation restrictions 
(i.e. unlimited post length, NSFW content, freeform gender 
presentation), to changes in default profile settings (user connections 
invisible to others) and significant demographic differences from other 
networks (tumblr has a 
notably\cite{drager2012trans,duggan2013demographics} higher 
percentage of young and LGBT users compared to other networks).  


What impact do these distringuishing characteristics have on the underlying 
structure 
of tumblr, and what constraints do the structure of tumblr impose on 
these characteristics?  We propose to begin to answer these questions by 
mining data from a subset of tumblr, and analyzing it with the following 
additional questions in mind.  What kind of posts reach prominence on 
tumblr?  Do users primarily act as content generators, or as content 
distributors?  What level of confidence can we reach in inferring 
(invisible) user-follows-user links by our knowledge of (visible)
user-\{likes|reblogs|posts\}-content links?

\end{abstract}
%%% Local Variables: 
%%% mode: latex
%%% TeX-master: "main"
%%% End: 
