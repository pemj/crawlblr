\begin{abstract}
This paper is chiefly concerned with the analysis of tumblr, an influential, 
yet mostly unexamined, online social network.  tumblr appears to exhibit 
several properties that make it qualitatively and quantitatively distinct 
from other popular online social networks (hereafter referred to as OSNs), 
such as Twitter, Facebook, Pinterest, and Instagram.  These differences 
range from the functional 
(unlimited post length, more types of embeddable content), to the social 
(tumblr has a notably\cite{drager2012trans,duggan2013demographics} higher 
percentage of young and LGBT users compared to other networks).  


Are these differences a result of some unknown social force?  Is tumblr 
simply unaccountably ``cooler'' than Twitter or Facebook? If this 
is not the case, can we trace the differences in usage patterns back to 
underlying differences in the functionality that tumblr offers its users?
We posit that an analysis of tumblr's emergent network properties is the 
first step to shedding some light on the effects of tumblr's functionality 
on its usage.  By crawling a large, connected subset of tumblr and carrying out several 
types of analysis, we conclude that tumblr offers a set of functionality 
distinct from other social networks.  We draw preliminary connections 
between social behavior unique to tumblr, and the topological/functional 
properties that make them possible.  


\end{abstract}
%%% Local Variables: 
%%% mode: latex
%%% TeX-master: "main"
%%% End: 
