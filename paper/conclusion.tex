\section{Conclusion}
\label{sec:-conc}
\subsection{Takeaway}

In this paper, we identified Tumblr as a promising subject for study 
by its comparatively novel position among other OSNs, and by the 
comparative dearth of analysis that it has thus far received despite 
that position.  In the course of this paper, we enumerated some of the 
differences between Tumblr and other networks, and provide a platform 
for future work in this field.  In so doing, we present some of the 
results generated by our analyses on a subset of the Tumblr network.


background
In Section \ref{sec:-back}, we attempted to place Tumblr within the 
broader context of previous research in OSNs.  By providing examples 
of the type of traffic flow within Facebook and Twitter, we explained 
the motivation behind

methodology


results


future work



\section{Future Work}
\label{sec:-fut}
We believe that this paper represents a significant stepping stone 
on the road to a better understanding of Tumblr and its place among 
other OSNs.  However, we acknowledge that this is still, in many ways, 
a prelude to further research.  With the work in this paper to serve as 
a guidepost, there are a multitude of avenues along which further 
research can be conducted.  We would like to enumerates some of these 
possibilities, should anyone have the opportunity to undertake further 
research along such lines.

Although we have operated on what we believe to be a significant 
representative subset of the Tumblr network, future studies would 
benefit from crawling a much greater breadth of the network.  Even 
something as simple as starting the study at different seed points
could lead to information regarding the connectivity (and potential 
completeness thereof) of Tumblr's user base.  Furthermore, this 
research could be greatly supplemented by the collection of similar 
slices of Tumblr separated by time, in order to investigate the manner 
in which the network evolves over time.  This could, for instances, 
tell us how long it takes for new users to become enmeshed to a 
significant degree in the network.


Early on in the project, we chose not to log post content during the 
running of the crawler.  This decision was largely influenced by the 
scope of our project, which focused primarily on metadata rather than 
on direct analysis of data.  However, small projects have extracted 
interesting results\cite{konczal2011parsing} by parsing the text of 
certain select subsets of Tumblr with known quantities.  It appears 
that Tumblr posts contain enough interesting information to warrant 
the examination through textual analysis, or even the application of 
machine learning in an effort to predict certain post/user 
characteristics based on observed content.  This may be as intuitive 
as classifying the preponderance of duplicate posts (original content 
entering the network from multiple disparate sources) or as complex as 
attempting to determine the age of a poster based on an analysis of 
profile data and post syntax.


In addition, while the quantity of data acquired by our crawler was 
effective for our limited time-frame for collection and analysis, we 
did not operate on one of the more complex features of Tumblr, the 
tagspace.  Tags not only share an explicit many-to-many relationship 
with posts, but also share an implicit many-to-many relationship with 
users, who may choose to follow posts of a particular tag.  In 
particular, this latter relationship may complicate the prototype we 
developed to detect user to user following relationships.  Future 
work on the subject would be greatly enhanced by the collection and 
analysis of the complex relationships between users, posts, and tags.


In fact, the prototype following-analyzer would benefit from a much 
larger focus in a future research projects.  The first version of ths 
analyzer suffered from an untenable time complexity 
(n\textsuperscript{2}m\textsuperscript{2} over posts and notes).
While the final version utilized in this paper was substantially 
more efficient, consultation with an expert in this field during the 
reconstruction of this algorithm could lead to analysis on a 
significantly larger dataset.  However, even if this algorithm was 
perfected within these bounds, it is still primarily a theoretical 
tool unless it is combined with a survey of users to engender feedback 
on the accuracy of our inference method.


This type of direct user consultation begins to enter the realm of 
research more frequently carried out by social scientists.  In fact, 
we believe that our analyses could benefit greatly from a 
cross-disciplinary approach to future work.



In order to enable future work on the subject of Tumblr, for ourselves 
as well as others, we have uploaded our code to a public github 
repository.


%%% Local Variables: 
%%% mode: latex
%%% TeX-master: "main"
%%% End: 
