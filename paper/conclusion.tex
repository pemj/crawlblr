\section{Conclusion and Future Work}
\label{sec:-conc}

We speculate that tumblr, in many ways, reflects the attitudes and 
needs of today's youth.  The anonymity of one's tumbl


There are several directions in which this project could be extended.


Although we have operated on a significant subset of tumblr, we
would like to carry out future studies on a greater breadth of the 
network.  A longer running study could yield a much more robust and 
complete picture of the network.

Early on in the project, we chose not to log post content during the 
running of the crawler.  In part, this was due to the time and space 
constraints imposed by the limited scope of the project, and the power 
of the system available for our research purposes.  In future work, 
time permitting, we could focus further on the textual similarities 
between posts.  Is brevity truly the soul of wit?

In addition, while our current database framework makes it easy to store 
our current data, it is not well-suited to the task of recording tags.  
Tags share a many-to-many relationship with posts, and could be more 
accurately modeled by a more robust object relational mapping.


In addition, we would like to observe the changes in the network from 
different snapshots of tumblr over a period of months.

We would like to send our inferred follower lists to a number of 
tumblr users and request some information about the correctness of our 
list, in order to refine our inference method.

As the structure of tumblr's namespace requires that our crawler be 
bootstrapped from a single hardcoded username, we would like to run our 
study with different starting points.

%%% Local Variables: 
%%% mode: latex
%%% TeX-master: "main"
%%% End: 
