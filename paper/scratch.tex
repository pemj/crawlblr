How is tumblr different from twitter?

First of all, how are they alike?

Superficially, their network shares nearly all of the same linkages.  
Users have the ability to contribute content to the network (post/tweet), 
retransmit content over the network (reblog/retweet), mark content for 
attention of other users (@user), signify their interest in content 
(like/favorite), and tag content (dedicated tagspace, inline hashtags).

When one begins to explore these areas of functionality further, we 
discover a number of implementation details that lead to divergence 
of form and topology.  The unrestricted length of tumblr posts leads to 
a different usage profile, users are likely to spend hours at the 
time\cite{duggan2013demographics} browsing their tumblr feeds.  

In addition, while both twitter and tumblr support multiple tags and 
user attention markers (#s and @s, respectively), twitter counts these 
markers against the character limit, restricting the degree of 
``multicast'' possible for a given post.  As tumblr has no restriction 
on message length, it allows for an arbitrary degree of multicast and 
sorting.  This means that users are far more likely to multiplex posts 
over multiple tags, creating a more comprehensive system of tag-based 
navigation, in which one can execute complex boolean search over posts.  
This has led to unanticipated emergent features. One of the most 
remarkable of these features leads to an increase in accessability for 
sufferers of post traumatic stress disorder.  It is considered good 
tumblr ettiquete to tag certain pieces of sensitive content with trigger 
warnings appropraite to the subject matter.  This allows users with 
posttraumatic stress disorder and other sufferers of psychological 
trauma to filter their experience, constructing a personal overlay 
which is devoid of content that may otherwise cause panic attacks or 
other negative psychological associations.


This may be connected with the demographic information of tumblr, which 
indicates a higher usage by women of color.  As a group, women of color 
from inner-city backgrounds in particular are more likely than US combat 
veterans to suffer the type of events recognized to cause PTSD.


This is also, however, a difficult question to answer.  While some 
social networks such as Twitter and Google+ seem nearly designed to 
expose information to the canny researcher, Tumblr maintains 
 
OSNs seem to be differentiated from earlier methods of communication by 
the differences in how connections are formed.  Bulletin Board Systems and
IRC chatrooms may contain much of the same content as OSNs, but that 
content is generally sorted by a different metric.  If one is interested in 
some content falling into a certain category, this system presumes that one is 
also interested in other content in that category.  OSNs, by contrast, 
generally sort content by its origin.  Instead, the system presumes that, 
if one has displayed interest in the content generated by some user, one is 
likely to display interest in other content generated by that same user.

From there, the relationship between users and content expands into a 
more intuitive social metaphor, linking not only users to content, but 
to each other.  In many ways, this is an extension of the model of the 
so-called blogosphere, in which one manually maintains connections to 
a content generation engine for any number of other users, forging 
implicit connections rather than the explicit connections offered by 
social networks. 



While it seems that the so-called blogosphere is entering a decline 
as users move to dedicated social media platforms, an analysis of the 
phenomena by the New York Times\cite{kopytoff2011blogs} indicates that 
the ``death of blogging'' is closer to an evolution.  As users move to 
OSNs such as Facebook, Twitter, and Tumblr, they
Tumblr is an odd duck.  Not only does it exhibit a demographic usage 
pattern when compared to other OSNs,

 in many ways it tries to unify 
the sort of simple wordpress blog experience that has been popular 
across the web, creating a sort of 

We then encode these notions into a query over the database and place 
the results in a hash for efficient analysis.  Having developed a 
prototypical method for identifying followers, we then show the 
results from some simple analyses carried out on that dataset.






%%% Local Variables: 
%%% mode: latex
%%% TeX-master: "main"
%%% End: 
