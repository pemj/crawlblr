How is tumblr different from twitter?

First of all, how are they alike?

Superficially, their network shares nearly all of the same linkages.  
Users have the ability to contribute content to the network (post/tweet), 
retransmit content over the network (reblog/retweet), mark content for 
attention of other users (@user), signify their interest in content 
(like/favorite), and tag content (dedicated tagspace, inline hashtags).

When one begins to explore these areas of functionality further, we 
discover a number of implementation details that lead to divergence 
of form and topology.  The unrestricted length of tumblr posts leads to 
a different usage profile, users are likely to spend hours at the 
time\cite{duggan2013demographics} browsing their tumblr feeds.  

In addition, while both twitter and tumblr support multiple tags and 
user attention markers (#s and @s, respectively), twitter counts these 
markers against the character limit, restricting the degree of 
``multicast'' possible for a given post.  As tumblr has no restriction 
on message length, it allows for an arbitrary degree of multicast and 
sorting.  This means that users are far more likely to multiplex posts 
over multiple tags, creating a more comprehensive system of tag-based 
navigation, in which one can execute complex boolean search over posts.  
This has led to unanticipated emergent features. One of the most 
remarkable of these features leads to an increase in accessability for 
sufferers of post traumatic stress disorder.  It is considered good 
tumblr ettiquete to tag certain pieces of sensitive content with trigger 
warnings appropraite to the subject matter.  This allows users with 
posttraumatic stress disorder and other sufferers of psychological 
trauma to filter their experience, constructing a personal overlay 
which is devoid of content that may otherwise cause panic attacks or 
other negative psychological associations.


This may be connected with the demographic information of tumblr, which 
indicates a higher usage by women of color.  As a group, women of color 
from inner-city backgrounds in particular are more likely than US combat 
veterans to suffer the type of events recognized to cause PTSD.





%%% Local Variables: 
%%% mode: latex
%%% TeX-master: "main"
%%% End: 
