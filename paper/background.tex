\section{Background}
\label{sec:-back}
\subsection{OSN Research}


Human interaction mediated by the internet presents a new and fascinating 
collection of data\cite{werry1996internet}.  As computer scientists, 
we suddenly have a view of social interaction that is amenable to study 
using the technical tools we are accustomed to using for the study of 
more traditionally rigorous subjects.  Unfortunately, these interactions 
have historically taken place over a set of wildly varied communication 
mediums and content platforms, creating a very heterogeneous collection 
of data that resists easy classification and analysis.  Therefore, while 
it may have been possible to study the collective user base of individual 
forms of communication\cite{reid1991electropolis}, the complete corpus of 
these interactions takes place at such differing levels of subtlety and 
complexity as to make that study extraordinarily difficult.

However, when the environment for such a series of interactions becomes 
flattened into a single, consistent topology, we are able to extract 
great quantities of data.  Online social networks provide extremely 
rich and explicitly contextual ecosystems, fertile planting grounds for 
this type of research.  Tumblr was founded in early 2007, contains over 
175 million separate blogs posting over 100 million total posts a 
day\cite{tumblr:about}, and was recently purchased by Yahoo for a 
rumored 1.1 billion dollars\cite{bbc-business}.  Despite its prominent 
position among OSNs, Tumblr has seen little study.  What studies do 
exist view Tumblr primarily through the lens of social sciences, rather than 
computer science.  From these studies, we can gain a rough picture of 
Tumblr's place within the broader context of OSNs, but no specifically 
targeted data analysis of any sort.  In the light of the quantity and 
quality of research that has been carried out in the service of studying 
OSNs, the marked lack of study on such a large set of data seems curious.


In order to identify the source of this research disparity, and
identify directions for study, we will briefly compare and contrast 
Tumblr to several other popular networks.  In so doing, we encounter 
several questions about the nature of Tumblr that we will later 
attempt to answer.


\subsection{Twitter}
\begin{figure}[bht]
\centering
 \includegraphics[width=3.0in]{twitter}
 \caption{An example of information flow over Twitter, from content Generators \(G\_1\) to content consumers/Distributors \(D\)}
 \label{fig:twitter}
\end{figure}
Twitter, in particular, has seen a massive flurry of research.  There 
are a number of reasons that it has been very popular with researchers.  
For one thing, Twitter is an extremely mature platform, as far as OSNs 
go.  This has resulted in a number of attractive characteristics, 
including but not limited to an ironclad API, a very large 
userbase\cite{krishnamurthy2008few}, 
and a comparatively long history that lends the enterprise a rich 
contextual basis for research.  In addition, the restriction on the 
heterogeneity and length of Twitter content makes it possible to carry 
out analysis of greater breadth and depth, respectively.


Superficially, their network shares nearly all of the same linkages.  
Users have the ability to contribute content to the network 
(post/tweet), re-transmit content over the network (reblog/retweet), 
mark content for attention of other users (@user), signify their 
interest in content (like/favorite), and tag content (dedicated 
tagspace, inline hashtags). When one begins to explore these areas of 
functionality further, we discover a number of implementation details 
that lead to divergence of form and topology.  

The unrestricted length of Tumblr posts leads to a different usage 
profile; users are likely to spend hours at a time\cite{fox2012much} 
browsing their Tumblr feeds, compared to just minutes a day in the case 
of Twitter.  In addition, while both Twitter and Tumblr support multiple 
tags and user attention markers (\#s and @s, respectively), Twitter 
counts these markers against the character limit, restricting the degree 
to which one can cross-reference a given post.  As Tumblr has no restriction 
on message length, it allows for an arbitrary degree of multicast and 
sorting.  This means that users are far more likely to cross-reference 
posts over multiple tags, creating a more comprehensive system of tag-based 
navigation, in which one can execute complex Boolean searches over posts.  


This has led to unanticipated emergent features. One of the most 
remarkable of these features has taken the form of an increase in 
accessability for sufferers of post-traumatic stress disorder, 
eating disorders\cite{callaghan2013research}, addiction, and other 
mental health issues.  It is considered good Tumblr etiquette to tag 
certain pieces of sensitive content with trigger 
warnings\cite{bell2013trigger} appropriate to the subject matter.  
This allows users to filter\cite{tumblrsavior} their OSN experience, 
constructing a personal overlay devoid of content that may otherwise 
cause panic attacks or harmful psychological episodes.


\subsection{Facebook}
\begin{figure}[bht]
\centering
 \includegraphics[width=3.0in]{facebook}
 \caption{An example of several different connections over Facebook: friendship, group membership, and liking of pages}
 \label{fig:facebook}
\end{figure}
Facebook's functionality can nearly be considered a superset of Twitter's.  
While the only user-user relationship in Twitter is the directional following 
relationship, Facebook users are more likely to interact using its 
bidirectional friendship relationship.  In addition to users and their 
posts, Facebook users may also interact with Pages and Groups.  
Similar to Twitter, Facebook has a very large dataset.  While attempts 
to study it may fall prey to the complexity resulting from the 
heterogeneity of its content, that same quality can present a much 
broader set of data if analyzed correctly.  Unlike Twitter, much of 
the data Facebook users generate is, on some level, private.  For 
better or for worse, users have a much greater range of options when 
selecting what pieces of information are shared among what portions 
of the network.


Facebook users are placed in control of outgoing information, 
and many studies have reflected this goal by directing their own 
research\cite{dwyer2007trust, lipford2008understanding}
towards the relationship that users perceive themselves to have with 
their own information, and the levels of trust they place in that 
network to protect the privacy of that data.  As Facebook requires 
information from its users to continue monetizing them, this places 
their OSN in a delicate balancing act.


By contrast, Tumblr is seen not as a place for the selective hiding of 
information\cite{liu2011analyzing}, but for explicitly sharing all of 
the information they care to.  Posts and profile information are 
inherently public, and users have no expectation of privacy.  This is 
reflected in the sparse profile information of Tumblr users: They are 
not required to provide and display a hometown, a real name, or even a 
specific gender.  Where Facebook users may use their account to keep 
up with family members\cite{joinson2008looking} or other geographically 
co-located users, the anonymization\cite{alexander2002introduction} 
inherent to Tumblr is more likely to lead to different patterns of 
usage by its userbase, and different demographic 
compositions\cite{drager2012trans} of that userbase.  


\subsection{The Nature of Tumblr}
Tumblr has captured a larger percentage of the teenage 
demographic\cite{glenn2013more} than any other OSN currently in popular use.  
Perhaps this has contributed to the view of Tumblr as an ``immature'' 
platform, compared to other networks.  Additionally, its API is 
notably less mature and featured when compared to, for example, 
Twitter.  Google+ was founded by a company that has literally made 
its money from cleverly associating users and content, and the 
structure of Google+ appears to have been constructed with that 
obsession with analysis in mind, even going so far as to collaborate 
with researchers carrying out studies\cite{kairam2012talking}.  By 
contrast, Tumblr was created independently by a small team, only to 
later be acquired by multinational internet corporation, Yahoo.

In the following section, we describe the tools we used to observe the 
structure of Tumblr's network.  In the course of the creation of these 
tools, we discover certain implementation details that may have had 
a hand in the lack of research over the network. 


%%% Local Variables: 
%%% mode: latex
%%% TeX-master: "main"
%%% End: 
