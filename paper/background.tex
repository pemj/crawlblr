\section{Background}
\label{sec:-back}
\subsection{OSN Research}


Human interaction over the internet has the potential to enable a great 
deal of research.  We suddenly have an open window into a tremendous 
amount of already quantified information for researchers to distill into 


hat does not require invasive experiments to study.  However, 
historically, these interactions have taken place in very different 
segments of the world wide web.  It is not feasible to aggregate the 
actions of a single person over their entire presence on the Internet.  
The largely heterogeneous nature of this traffic prevents the easy 
classification of data.  

However, when the environment for such a series 
of interactions becomes flattened into a single, consistent topology, 
we are able to extract great quantities of data.  Online social 
networks are capable of providing such ecosystems.  Oddly enough, 
for such a popular network, tumblr has seen little study.  What studies 
do exist originate primarily from the social sciences, rather than computer 
science. 


 



While it seems that the so-called blogosphere is entering a decline 
as users move to dedicated social media platforms, an analysis of the 
phenomena by the New York Times\cite{kopytoff2011blogs} indicates that 
the ``death of blogging'' is closer to an evolution.  As users move to 
OSNs such as Facebook, Twitter, and Tumblr, they
tumblr is an odd duck.  Not only does it exhibit a demographic usage 
pattern when compared to other OSNs,

 in many ways it tries to unify 
the sort of simple wordpress blog experience that has been popular 
across the web, creating a sort of 



Many popular online social networks coexist without necessarily 
impinging on each other's markets shares.  This is possible because 
different OSNs fill difference niches in the social interactions 
between users.  In the light of the quantity and quality of research 
that has been carried out in the service of studying OSNs such as 
Twitter, the question of whether or not such illuminating results 
can be extracted from other networks is an intriguing one.  

This is also, however, a difficult question to answer.  While some 
social networks such as Twitter and Google+ seem nearly designed to 
expose information to the canny researcher, tumblr maintains 



OSNs seem to be differentiated from earlier methods of communication by 
the differences in how connections are formed.  Bulletin Board Systems and
IRC chatrooms may contain much of the same content as OSNs, but that 
content is generally sorted by a different metric.  If one is interested in 
some content falling into a certain category, this system presumes that one is 
also interested in other content in that category.  OSNs, by contrast, 
generally sort content by its origin.  Instead, the system presumes that, 
if one has displayed interest in the content generated by some user, one is 
likely to display interest in other content generated by that same user.

From there, the relationship between users and content expands into a 
more intuitive social metaphor, linking not only users to content, but 
to each other.  In many ways, this is an extension of the model of the 
so-called blogosphere, in which one manually maintains connections to 
a content generation engine for any number of other users, forging 
implicit connections rather than the explicit connections offered by 
social networks. 




 We contrast tumblr with the following networks, in order to classify 
what makes tumblr distinct enough to warrant research, and what has 
made other networks more popular with research.

\subsection{Twitter}
\begin{figure}[bht]
\centering
 \includegraphics[width=3.0in]{twitter}
 \caption{An example of information flow over Twitter, from content Generators \(G\_1\) to content consumers/Distributors \(D\)}
 \label{fig:twitter}
\end{figure}
Twitter, in particular, has seen a massive flurry of research.  There 
are a number of reasons that it has been very popular with researchers.  
For one thing, Twitter is an extremely mature platform, as far as OSNs 
go.  This has resulted in a number of attractive characteristics, 
including but not limited to an ironclad API, a very large userbase, 
and a comparatively long history that lends the enterprise a rich 
contextual basis for research.  In addition, the restriction on the 
heterogeneity and length of Twitter content makes it possible to carry 
out analysis of greater breadth and depth, respectively.


Superficially, their network shares nearly all of the same linkages.  
Users have the ability to contribute content to the network 
(post/tweet),re-transmit content over the network (reblog/retweet), 
mark content for attention of other users (@user), signify their 
interest in content (like/favorite), and tag content (dedicated 
tagspace, inline hashtags). When one begins to explore these areas of 
functionality further, we discover a number of implementation details 
that lead to divergence of form and topology.  The unrestricted length 
of tumblr posts leads to a different usage profile; users are likely to 
spend hours at a time\cite{duggan2013demographics} browsing their 
tumblr feeds.  


In addition, while both Twitter and tumblr support multiple tags and 
user attention markers (\#s and @s, respectively), Twitter counts these 
markers against the character limit, restricting the degree of 
``multicast'' possible for a given post.  As tumblr has no restriction 
on message length, it allows for an arbitrary degree of multicast and 
sorting.  This means that users are far more likely to cross-reference 
posts over multiple tags, creating a more comprehensive system of tag-based 
navigation, in which one can execute complex Boolean searches over posts.  
This has led to unanticipated emergent features. One of the most 
remarkable of these features leads to an increase in accessability for 
sufferers of post traumatic stress disorder.  It is considered good 
tumblr etiquette to tag certain pieces of sensitive content with trigger 
warnings appropriate to the subject matter.  This allows users with 
posttraumatic stress disorder and other sufferers of psychological 
trauma to filter their experience, constructing a personal overlay 
which is devoid of content that may otherwise cause panic attacks or 
other negative psychological associations.


\subsection{Facebook}
\begin{figure}[bht]
\centering
 \includegraphics[width=3.0in]{facebook}
 \caption{An example of several different connections over Facebook: friendship, group membership, and liking of pages}
 \label{fig:facebook}
\end{figure}
Facebook's functionality can nearly be considered a superset of Twitter's.  
While the 
only user-user relationship in Twitter is the directional following 
relationship, Facebook users are more likely to interact using its 
bidirectional friendship relationship.  In addition to users and their 
posts, Facebook users may also interact with Pages and Groups.  Unlike 
Twitter, many Facebook conversations are private, and the users have a 
greater degree of control over what pieces of information are shared 
with each other user.  But while Facebook users are placed in control of 
outgoing information, tumblr users place a great deal of stock in their 
ability to control incoming information, as discussed above.

 
Similar to Twitter, Facebook has a very large dataset.  While attempts 
to study it may fall prey to the complexity resulting from the 
heterogeneity of its content, that same quality can present a much 
broader set of data if analyzed correctly.

\subsection{Demographics}
tumblr has a larger usage base in the teenage demographic than any 
other OSN currently in use.  Perhaps this has contributed to the view 
of tumblr as an ``immature'' subject for research, compared to other 
networks.  Additionally, its API is notably less mature and 
featured when compared to, for example, Twitter.  Google+ was founded 
by a company that has literally made its money from cleverly 
associating users and content, and built its network with that 
obsession with analysis in mind, even going so far as to collaborate 
with researchers carrying out studies\cite{kairam2012talking}.  By contrast, tumblr was 
created independently before being taken over by Yahoo, leading to an 
infrastructure less well-suited for study

\cite{alexander2002introduction}

We observe that a closed environment in the context
%%% Local Variables: 
%%% mode: latex
%%% TeX-master: "main"
%%% End: 
