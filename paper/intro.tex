\section{Introduction}
\label{sec:-intro}


Historically, OSNs have benefited from an economy of scale.  As users 
flock to a network, it becomes easier for them to find their friends, 
and it becomes easier for the network to generate ad revenue.  However, 
if this were more than a back-of-the-napkin sort of rule, we would 
expect a single killer-app of social networking to reach 100\% market 
penetration.  Instead, a variety of large social networks continue 
to exist side by side.  Users seem to prefer a series of different 
networks, with different sets of functionality.  While we cannot 
outright discount the possibility that purely social forces have led to the 
propagation of different networks, previous studies on the structures 
of OSNs such as Twitter\cite{kwak2010twitter} and Facebook\cite{viswanath2009evolution}
seem to indicate that the functional structure of the content 
distribution/interaction system has a significant influence on what set of 
users is likely to participate in that system, as well as how those users 
are likely to participate in that system.


With that assumption in mind, we posit that Tumblr's position as a 
comparatively popular OSN may partially reflect some 
set of functional differences from other social networks.  However, to 
the best of our knowledge, Tumblr has not yet been the subject of 
a rigorous research project investigating its structural 
and functional properties.  We propose to carry out an exploratory study, 
investigating several properties of Tumblr, positing several preliminary 
theories as to the effects of these properties, and developing a reusable 
model for representing and collecting information from Tumblr while so 
doing.

As a consequence of Tumblr's structure, we found that the process of 
data collection was best enabled by the construction of a 
Tumblr-specific web crawler.  This web crawler was designed to gather 
as much information as was feasible and analyzable within the time-frame 
dictated by the nature of the project.  We present a more complete 
outline of this process in Section \ref{sec:-method}.  We constructed 
our crawler, and deployed it onto a node of the ACISS supercomputing 
cluster at the University of Oregon.  We gathered data in one 24-hr 
round, and two 48-hr rounds, and compiled the data thus gathered into a 
22-gigabyte relational database.  This information was primarily arranged 
as nodes representing users and posts, and edges relating users to 
posts in various ways. Our research questions were then translated into 
queries on that database, also carried out using an ACISS node.

We present the results of our queries, and a preliminary analysis of 
those results, in Section \ref{sec:-res}.  We present a collection of 
data related to the relative frequency with which users reblog 
content from others, compared to the frequency with which they create 
their own content.  We find that reblogging occurs much more frequently, 
and that many users generate very little original content.  Of both 
reblogged and original content, we measure which type is the most 
popular, and find that text posts are at least an order of magnitude 
more popular than other posts.  In addition, we outline the development 
of a method for classifying user->user relationships, based on available 
user->content relationship data.  While this method is computationally 
expensive over larger datasets, we believe it represents a promising 
first step in terms of inferring following relationships from like 
and reblog relations.


In Section \ref{sec:-conc}, we consider some implications of the data 
we have gathered.  Given that Tumblr offers a different set of 
functionality, what are some potential ways that these differences may 
have driven a different user base to Tumblr, and how might that have 
fed back into the sort of interactions that take place there?  To 
answer these questions, and others, requires both a deeper and broader 
approach than our own.  In the future, we would be interested not only 
in collecting a more complete set of data, but analyzing that data 
alongside specialists in the social sciences.  We believe this 
multidisciplinary approach to be well-suited for the analysis of the 
technological and social phenomena involved.







%%% Local Variables: 
%%% mode: latex
%%% TeX-master: "main"
%%% End: 
