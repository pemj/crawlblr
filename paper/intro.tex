\section{Introduction}
\label{sec:-intro}

OSNs clearly benefit from an economy of scale.  As users flock to a 
network, it becomes easier for them to find their friends, and it 
becomes easier for the network to generate ad revenue.  Why, then, 
do users continue not only to sign up for different networks from their 
friends, but also to multiple networks simultaneously?  How do we explain the 
glaring absence of a single, all-powerful OSN, one killer app for the 
social networking age?  Obviously, not all OSNs are created equal.  
Facebook certainly seems to be trying to structure their service 
as an Internet overlay providing all necessary services (news, video, 
music, games, messaging, employment, to enumerate a few), and may even 
be expanding into the aerospace business in order to offer their 
services to areas of the developing world without access to traditional 
landline-based internet connectivity.  And yet, users seem to prefer 
a series of compartmentalized environments optimized for different 
usage profiles.

OSNs entered the picture on the heels of the blogosphere, which became 
one of the dominant modes of online communication, themselves 
successors to the BBS scene.  As technology enables new ways of 
relating with others, the older methods have largely fallen by the 
wayside.  Or have they?  While Myspace, Facebook, and Twitter professed 
to be entirely new ways of relating to people, tumblr emphasizes the 
familiar, quick solution.  Why, they ask, should one go through the 
trouble to arrange for an account on Blogspot, Blogger, or Wordpress 
when you're only thirty seconds away from your first post, the 
beginning of the Great American Novel?


Each generation has its media scare, the new subject that parents 
never had to deal with when they were kids, and that they now find 
themselves in a position to regulate their children's access to said 
subject.  We have run the gamut on moral panic followed by eventual 
settling down into the new order.  Where once parents may be concerned 
about their children's behavior after reading violent comic books, many 
parents now find themselves in a position to
violent television, violent videogames, the unrestricted and 
intimidatingly public world of internet blogging.

While it seems that the so-called blogosphere is entering a decline 
as users move to dedicated social media platforms, an analysis of the 
phenomena by the New York Times\cite{kopytoff2011blogs} indicates that 
the ``death of blogging'' is closer to an evolution.  As users move to 
OSNs such as Facebook, Twitter, and Tumblr, they
tumblr is an odd duck.  Not only does it exhibit a demographic usage 
pattern when compared to other OSNs,

 in many ways it tries to unify 
the sort of simple wordpress blog experience that has been popular 
across the web, creating a sort of 



Many popular online social networks coexist without necessarily 
impinging on each other's markets shares.  This is possible because 
different OSNs fill difference niches in the social interactions 
between users.  In the light of the quantity and quality of research 
that has been carried out in the service of studying OSNs such as 
Twitter, the question of whether or not such illuminating results 
can be extracted from other networks is an intriguing one.  

This is also, however, a difficult question to answer.  While some 
social networks such as Twitter and Google+ seem nearly designed to 
expose information to the canny researcher, tumblr maintains 





%%% Local Variables: 
%%% mode: latex
%%% TeX-master: "main"
%%% End: 
